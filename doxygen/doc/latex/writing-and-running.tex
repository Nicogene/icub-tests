\hypertarget{writing-and-running_writing-new-tests}{}\section{Writing a new test}\label{writing-and-running_writing-new-tests}
Create a folder with the name of your test case in the {\ttfamily icub-\/tests/src/} folder to keep your test codes\+:


\begin{DoxyCode}
$ mkdir icub-tests/src/example-test
\end{DoxyCode}


Create a child test class inherited from the {\ttfamily Yarp\+Test\+Case}\+:


\begin{DoxyCode}
\textcolor{preprocessor}{#ifndef \_EXAMPLE\_TEST\_H\_}
\textcolor{preprocessor}{#define \_EXAMPLE\_TEST\_H\_}

\textcolor{preprocessor}{#include <YarpTestCase.h>}

\textcolor{keyword}{class }ExampleTest : \textcolor{keyword}{public} YarpTestCase \{
\textcolor{keyword}{public}:
    ExampleTest();
    \textcolor{keyword}{virtual} ~ExampleTest();

    \textcolor{keyword}{virtual} \textcolor{keywordtype}{bool} setup(yarp::os::Property& property);

    \textcolor{keyword}{virtual} \textcolor{keywordtype}{void} tearDown();

    \textcolor{keyword}{virtual} \textcolor{keywordtype}{void} run();
\};

\textcolor{preprocessor}{#endif //\_EXAMPLE\_TEST\_H}
\end{DoxyCode}


Implement the test case\+:


\begin{DoxyCode}
\textcolor{preprocessor}{#include <Plugin.h>}
\textcolor{preprocessor}{#include "ExampleTest.h"}

\textcolor{keyword}{using namespace }std;
\textcolor{keyword}{using namespace }RTF;
\textcolor{keyword}{using namespace }yarp::os;

\textcolor{comment}{// prepare the plugin}
PREPARE\_PLUGIN(ExampleTest)

ExampleTest::ExampleTest() : YarpTestCase("ExampleTest") \{
\}

ExampleTest::~ExampleTest() \{ \}

\textcolor{keywordtype}{bool} ExampleTest::setup(yarp::os::Property &property) \{

    \textcolor{comment}{// initialization goes here ...}
    \textcolor{comment}{//updating the test name}
    \textcolor{keywordflow}{if}(property.check(\textcolor{stringliteral}{"name"}))
        setName(property.find(\textcolor{stringliteral}{"name"}).asString());

    \textcolor{keywordtype}{string} example = \textcolor{keyword}{property}.check(\textcolor{stringliteral}{"example"}, Value(\textcolor{stringliteral}{"default value"})).asString();

    RTF\_TEST\_REPORT(Asserter::format(\textcolor{stringliteral}{"Use '%s' for the example param!"},
                                       example.c\_str()));
    \textcolor{keywordflow}{return} \textcolor{keyword}{true};
\}

\textcolor{keywordtype}{void} ExampleTest::tearDown() \{
    \textcolor{comment}{// finalization goes here ...}
\}

\textcolor{keywordtype}{void} ExampleTest::run() \{

    \textcolor{keywordtype}{int} a = 5; \textcolor{keywordtype}{int} b = 3;
    RTF\_TEST\_REPORT(\textcolor{stringliteral}{"testing a < b"});
    RTF\_TEST\_CHECK(a<b, Asserter::format(\textcolor{stringliteral}{"%d is not smaller than %d."}, a, b));
    RTF\_TEST\_REPORT(\textcolor{stringliteral}{"testing a > b"});
    RTF\_TEST\_CHECK(a>b, Asserter::format(\textcolor{stringliteral}{"%d is not smaller than %d."}, a, b));
    RTF\_TEST\_REPORT(\textcolor{stringliteral}{"testing a == b"});
    RTF\_TEST\_CHECK(a==b, Asserter::format(\textcolor{stringliteral}{"%d is not smaller than %d."}, a, b));
    \textcolor{comment}{// add more }
    \textcolor{comment}{// ...}
\}
\end{DoxyCode}


Notice\+: The {\ttfamily R\+T\+F\+\_\+\+T\+E\+S\+T\+\_\+\+C\+H\+E\+CK}, {\ttfamily R\+T\+F\+\_\+\+T\+E\+S\+T\+\_\+\+R\+E\+P\+O\+RT} do N\+OT threw any exception and are used to add failure or report messages to the result collector. Instead, all the macros which include {\ttfamily \+\_\+\+A\+S\+S\+E\+R\+T\+\_\+} within their names (e.\+g., {\ttfamily R\+T\+F\+\_\+\+A\+S\+S\+E\+R\+T\+\_\+\+F\+A\+IL}) throw exceptions which prevent only the current test case (Not the whole test suite) of being proceed. The error/failure messages thrown by the exceptions are caught. (See \href{http://robotology.github.io/robot-testing/documentation/TestAssert_8h.html}{\tt {\itshape Basic Assertion macros}}).

The report/assertion macros store the source line number where the check/report or assertion happen. To see them, you can run the test case or suit with {\ttfamily -\/-\/detail} parameter using the {\ttfamily testrunner} (See \href{http://robotology.github.io/robot-testing/documentation/testrunner.html}{\tt {\itshape Running test case plug-\/ins using testrunner}}).

Create a cmake file to build the plug-\/in\+:


\begin{DoxyCode}
cmake\_minimum\_required(VERSION 2.8.9)

\textcolor{preprocessor}{# set the project name}
\textcolor{keyword}{set}(PROJECTNAME ExampleTest)
project($\{PROJECTNAME\})

# add the required cmake packages
find\_package(RTF)
find\_package(RTF COMPONENTS DLL)
find\_package(YARP)

\textcolor{preprocessor}{# add include directories}
include\_directories($\{CMAKE\_SOURCE\_DIR\}
                    $\{RTF\_INCLUDE\_DIRS\}
                    $\{YARP\_INCLUDE\_DIRS\}
                    $\{YARP\_HELPERS\_INCLUDE\_DIR\})

# add required libraries 
link\_libraries($\{RTF\_LIBRARIES\}
               $\{YARP\_LIBRARIES\})

# add the source codes to build the plugin library
add\_library($\{PROJECTNAME\} MODULE ExampleTest.h
                                  ExampleTest.cpp)

# \textcolor{keyword}{set} the installation options
install(TARGETS $\{PROJECTNAME\}
        EXPORT $\{PROJECTNAME\}
        COMPONENT runtime
        LIBRARY DESTINATION lib)
\end{DoxyCode}


Call your cmake file from the {\ttfamily icub-\/test/\+C\+Make\+Lists.\+txt} to build it along with the other other test plugins. To do that, adds the following line to the {\ttfamily icub-\/test/\+C\+Make\+Lists.\+txt}


\begin{DoxyCode}
\textcolor{preprocessor}{# Build example test }
\textcolor{preprocessor}{add\_subdirectory(src/example-test)}
\end{DoxyCode}


Please check the {\ttfamily icub-\/tests/example} folder for a template for developing tests for the i\+Cub.\hypertarget{writing-and-running_running_single_test_case}{}\section{Running a single test case}\label{writing-and-running_running_single_test_case}
As it is documented here (\href{http://robotology.github.io/robot-testing/documentation/testrunner.html}{\tt {\itshape Running test case plug-\/ins using testrunner}}) you can run a single test case or run it with the other tests using a test suite. For example, to run a single test case\+:


\begin{DoxyCode}
testrunner --verbose --test plugins/ExampleTest.so  --param \textcolor{stringliteral}{"--name MyExampleTest"}
\end{DoxyCode}


Notice that this test require the {\ttfamily yarpserver} to be running and it contains tests that are programmed to succeed and some that are programmed to fail.

or to run the i\+Cub\+Sim camera test whith the test configuration file\+:


\begin{DoxyCode}
testrunner --verbose --test plugins/CameraTest.so --param \textcolor{stringliteral}{"--from camera\_right.ini"} --environment \textcolor{stringliteral}{"
      --robotname icubSim"}
\end{DoxyCode}


This runs the icub\+Sim right-\/camera test with the parameters specified in the {\ttfamily right\+\_\+camera.\+ini} which can be found in {\ttfamily icub-\/tests/suits/contexts/icub\+Sim} folder. This test assumes you are running {\ttfamily yarpserver} and the i\+Cub simulator (i.\+e. {\ttfamily i\+Cub\+\_\+\+S\+IM}).

Notice that the environment parameter {\ttfamily -\/-\/robotname icub\+Sim} is used to locate the correct context (for this examples is {\ttfamily icub\+Sim}) and also to update the variables loaded from the {\ttfamily right\+\_\+camera.\+ini} file.\hypertarget{writing-and-running_running_multiple_tests}{}\section{Running multiple tests using a test suite}\label{writing-and-running_running_multiple_tests}
You can update one of the existing suite X\+ML files to add your test case plug-\/in and its parameters or create a new test suite which keeps all the relevant test cases. For example the {\ttfamily basic-\/icub\+Sim.\+xml} test suite keeps the basic tests for cameras and motors\+:


\begin{DoxyCode}
<?xml version=\textcolor{stringliteral}{"1.0"} encoding=\textcolor{stringliteral}{"UTF-8"}?>

<suit name=\textcolor{stringliteral}{"Basic Tests Suite"}>
    <description>Testing robot\textcolor{stringliteral}{'s basic features</description>}
\textcolor{stringliteral}{    <environment>--robotname icubSim</environment>}
\textcolor{stringliteral}{    <fixture param="--fixture icubsim-fixture.xml"> yarpmanager </fixture>}
\textcolor{stringliteral}{}
\textcolor{stringliteral}{    }
\textcolor{stringliteral}{    <test type="dll" param="--from right\_camera.ini"> CameraTest </test>}
\textcolor{stringliteral}{    <test type="dll" param="--from left\_camera.ini"> CameraTest </test> }
\textcolor{stringliteral}{}
\textcolor{stringliteral}{    }
\textcolor{stringliteral}{    <test type="dll" param="--from test\_right\_arm.ini"> MotorTest </test>}
\textcolor{stringliteral}{    <test type="dll" param="--from test\_left\_arm.ini"> MotorTest </test>}
\textcolor{stringliteral}{</suit>}
\end{DoxyCode}


Then you can run all the test cases from the test suite\+:


\begin{DoxyCode}
testrunner --verbose --suit icub-tests/suits/basics-icubSim.xml
\end{DoxyCode}


The {\ttfamily testrunner}, first, launches the i\+Cub simulator and then runs all the tests one after each other. After running all the test cases, the {\ttfamily tesrunner} stop the simulator. If the i\+Cub simulator crashes during the test run, the {\ttfamily testrunner} re-\/launchs it and continues running the remaining tests.

How {\ttfamily testrunner} knows that it should launch the i\+Cub simulator before running the tests? Well, this is indicated by {\ttfamily $<$fixture param=\char`\"{}-\/-\/fixture icubsim-\/fixture.\+xml\char`\"{}$>$ yarpmanager $<$/fixture$>$}. The {\ttfamily testrunner} uses the {\ttfamily yarpmanager} fixture plug-\/in to launch the modules which are listed in the {\ttfamily icubsim-\/fixture.\+xml}. Notice that all the fixture files should be located in the {\ttfamily icub-\/tests/suits/fixtures} folder. 